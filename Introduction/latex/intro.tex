\documentclass[UTF8]{ctexart}

\usepackage{hyperref}
\usepackage{xcolor}

\usepackage{xeCJKfntef}

\title{``让我们一起识破商家的小伎俩!''作品介绍}
\author{张翼林}
\date{\today}

\begin{document}

\maketitle

作品开源在\href{https://github.com/aiCane/taobao-price-simulator}{\textit{GitHub}}上,可以clone下来,在本地体验。我也在\textit{Streamlit Community}上部署了网页,可以通过点击\href{https://taobao-price-simulator-0-1-1.streamlit.app"}{「这里」}来体验,不过有些时候它会down掉,不是很好用。我还是推荐部署在本地使用。

\section*{灵感来源}

我本来不知道去做一个怎样的作业,一直拖拖拖着。直到有次朋友的耳机丢了,她非常地懊恼,痛定思痛之后决定买一个一模一样的旧款耳机。然后她跑过来抱怨说:\textbf{``太贵了!''}

正好,我那时候在买自己的东西,发现有一个满500减60的满减,但是总差两百多凑不满。我向她询问了耳机的型号,想着凑个满减。

于是她发来了这张截图

\begin{center}
\textit{「一张截图」}
\end{center}

我拿着这张截图在淘宝找了找,找到的耳机却要便宜很多。那干嘛不买这个更便宜的呢?我将包含着耳机价格的购物车截图发给了她。

\begin{center}
\textit{「老师说文字稿不带图片」}
\end{center}

她说:"不信!"向我要来了同款链接,这下可以便宜买了。但是定睛一看,这一模一样链接在手里居然还贵了好多钱!

\begin{center}
\textit{「所以我把图片全删了」}
\end{center}

后来,我去研究为什么会出现如此严重的差异,找到了相关活动的规则界面《2025双旦消费券活动规则》。其中第三点中说

\bigskip
\begin{figure}[!h]
\centering
\begin{minipage}{0.6\linewidth}
\color{gray}
\small

``三、参与条件

本活动采用\CJKunderline{邀请制},用户须同时满足下述全部条件方\CJKunderline{有机会}参与本活动:

......

(三)不得为异常用户,异常用户判定标准详见[六、注意事项]第(三)条。

......''
\end{minipage}
\end{figure}
\bigskip

我注意到这其中有一个\textbf{异常用户}的界定,它在注意事项中:

\bigskip
\begin{figure}[!h]
\centering
\begin{minipage}{0.6\linewidth}
\color{gray}
\small

``(三)在用户参加本活动或使用本活动发放的优惠权益的过程中,\CJKunderline{如用户出现、经平台合理怀疑存在、或曾经存在违背诚实信用原则等非正常行为,将被视为异常用户,用户将可能面临被取消活动参与资格、无法获取权益、已获取的权益使用受到全部/部分限制等情形,必要时平台将追究用户的法律责任。}'非正常行为'包括但不限于:通过任何不正当手段达成交易或参与活动,如使用插件、外挂等工具或借助非自然流量或非正常社交关系完成好友助力任务(包括不限于使用专业助力群、交易获得助力);从事虚假交易;盗用身份、提供虚假信息;套取活动道具、套取补贴;从事赌博、洗钱、违规套现、刷信誉等行为;利用技术漏洞或规则漏洞下单、获取福利、补贴;违反平台协议。\CJKunderline{用户授权并同意,平台有权通过风控系统对用户是否存在前述非正常行为进行判定,有权依据用户注册/登录信息用户行为、普通人的正常交易习惯、互联网领域行业惯例、生活常识等大数据进行综合分析评估是否为异常用户。}'']
\end{minipage}
\end{figure}
\bigskip

那通过这个注意事项,就可以解释为要么是我运气好,被选中了参加活动,而我朋友没有?要么是她被界定为异常用户了?

于是期末作业的灵感就来了。一般人如果不去比较、做深究的话,可能\textbf{很难}发现电商居然做得这么狠!网警正义出击,我要去揭露这其中的秘密!\CJKsout{\small 虽然其实这个灵感与后文的``价格歧视''并不沾边。}

\section*{我使用的制作工具、过程}

我的核心思路是模拟一个``定价黑箱''。我先设计了几个能代表``用户画像''的选项,然后给每个选项分配不同的``权重值'',组合起来,再从一个基础价格开始往上加价或者打折。这样就能生成看起来不一样的价格了。

\begin{itemize}
  \item 整个网页都是用\textit{Streamlit}搭的,这个框架由于基于Python,对我这样的新手特别友好。
  \item 代码全是\textit{Python}写的,逻辑不复杂,主要功夫花在怎么把"用户画像"和"价格"编得合理上。
\end{itemize}

有了灵感,搭好架子,我就拿着这个半成品到处找人``采访''。我问他们:``现在的用户画像选项是不是有点少?你觉得还有哪些地方能再加什么?''然后把他们的想法做进选项里。又比如:``这里加个对比图?''``这个按钮放这儿顺手吗?'',就这样问一圈,改一遍,再问一圈,再改一遍...最后就成了现在这个样子。

\section*{作品的创意点、价格歧视和我的一些感想}

我想通过一个简单的网页,以一个类似于游戏的方式,去尽量模仿这个``不同人看到不同价''的现象。``大数据杀熟''这种词,大家平时都听过,但自己很难有切身体会。这个网页就想提供一个直观的感受,你点点选项,假装自己是不同的人,就能看到电商可能给你报出不同的价格。

简单来说:这种现象在经济学上被称为 \textbf{价格歧视}。它是指商家根据消费者的支付意愿、消费能力等特征,对相同的商品或服务制定不同价格的策略。

琼·罗宾逊在《不完全竞争经济学》中系统阐述了她的价格歧视理论,将价格歧视分为一级、二级和三级这三种类型:
\cite{schumpeter1934}

\begin{itemize}
  \item 一级价格歧视:厂商对每一单位产品均按消费者的最高支付意愿定价,从而攫取全部消费者剩余。
  \item 二级价格歧视:根据购买量或消费行为差异制定不同价格。
  \item 三级价格歧视:根据不同市场或消费者群体的需求弹性差异定价。
\end{itemize}

\textbf{算法价格歧视}这个词有不大一样了,它同时来源于算法和价格歧视,也同时拥有市场与技术特点。算法价格歧视是指企业借助算法技术,基于用户的行为数据(例如浏览历史、消费偏好、设备类型等),对同一商品或服务向不同消费者设定不同价格。其本质在于依托数据分析实现个性化定价,不再局限于传统价格歧视中的粗粒度群体划分,而能接近``一对一''的精准定价。
\cite{chen2025}

算法歧视的形成大致可以分为三个步骤:
\cite{li2021}

\begin{enumerate}
  \item 采集模型对消费者的个人信息进行的收集,构成算法歧视的基础:
  \begin{itemize}
    \item 以经营者的\textit{营利性目的}作为内在驱动力
    \item 对消费者的个人信息的收集是有\textit{选择性}的
    \item 收集包括经济状况、支付能力、支付意愿、消费场景等信息
  \end{itemize}
  \item 特定推送算法基于强大的数据清洗和处理能力,在了解消费者支付意愿后,将商品信息推送给特定标签群体或消费者,以实现用户获取有效信息的成本最小化和互联网企业利润最大化。
  \item 算法幕后的设计者或控制人可能滥用算法技术权力。如通过算法歧视进行个性化差异定价,快速谋取垄断利润。
\end{enumerate}

而这是有问题的。每个步骤都能找到猫腻,比如算法对巧合关联和真正的因果关系就无法加以区分,这就会导致错误。
\cite{yang2018}

我这个模拟器很简单,它里面的价格都是编的,和真实复杂的电商算法比就是小儿科。但它成功地让我的朋友们都觉得有点意思,引起了思考,那我觉得,这个作业的目的就达到了。

\bibliographystyle{plain}
\bibliography{refs}

\end{document}
